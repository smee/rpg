% getting_started.tex

\chapter{Getting Started}
This chapter provides a short real-world example of performing symbolic regression with \RGP.
\lipsum[1]


\section{Case Study Design and Organization}
TODO general overview, concepts of problem sheet, problem instance, and method, intro to
experimental desgin and planning via SPO

\lipsum[2]


\section{Case Studies in Finance}
Although the application of GP in finance is a young field, there already exist a comparatively
large body of literature on this subject (TODO lit). The largest fraction of these work applies GP
to evolve trading strategies for algorithmic trading. Our case studies are no exception in this
regard, but are special in the fact that they apply the same GP framework in an array of vastly
markets. TODO short intro to the three markets, etc.

\subsection{Algorithmic Trading in Stock Markets {\sf (AppStock)}}
TODO

\begin{table}[ht]
  \caption{Summary Information on the Stocks included in the Case Study. This Summary Information was retreived from Yahoo Finance on March 1, 2010.}
  \begin{tabular}{lll}
    \toprule
    {\em Symbol} & {\em Company} & {\em Sector}    \\
    \midrule
    {\sf AAPL}   & Apple Inc.    & Technology      \\
    {\sf CVX}    & Chevron Corp. & Basic Materials \\
    {\sf GOOG}   & Google Inc.   & Technology      \\
    {\sf PEP}    & Pepsico Inc.  & Consumer Goods  \\
    \bottomrule
  \end{tabular}
\end{table}

\lipsum[3]

\subsection{Algorithmic Trading  in the FX Market {\sf (AppFX)}}
TODO

\paragraph{Objective Function}
The vTrader fitness function $\mathit{fitness_{vt}}$ assigns a numerical fitness to a signal
generator function $s$. It is defined as follows:

\begin{align}
  \mathit{fitness_{vt}}(s) &:= [\underbrace{-1 \cdot \mathit{pnl}(s, \mathit{T_{train}})}_\text{performance term} %
    + \underbrace{\mathit{c_{mdd}} \cdot \mathit{mdd}(s, \mathit{T_{train}})] %
    \cdot \mathit{\mathit{balance}}(s)}_\text{penalty terms} \\
  \mathit{balance}(s)      &:= %
    \begin{cases}
      \mathit{c_{balance}} & \text{if } \mathit{skewness}(s) > \mathit{limit_{skewness}} \\
      1  & \text{otherwise}
    \end{cases} \\
  \mathit{skewness}(s)     &:= \frac{\mathit{abs}(|s(\mathit{T_{train}})|_{\mathit{long}} %
    - |s(\mathit{T_{train}})|_{\mathit{short}})}{|s(\mathit{T_{train}})|_{\mathit{all}}}
\end{align}

\begin{table}[ht]
  \caption{Default Parameter Values for the vTrader fitness function $\mathit{fitness_{vt}}$ used in the Case Study.}
  \begin{tabular}{lll}
    \toprule
    {\em Parameter}             & {\em Description}                  & {\em Value} \\
    \midrule
    $\mathit{c_{mdd}}$          & Max-Drawdown Penalty Factor        & TODO        \\
    $\mathit{limit_{skewness}}$ & Long/Short-Skewness Limit          & $0.8$       \\
    $\mathit{c_{balance}}$      & Long/Short-Skewness Penalty Factor & TODO        \\
    \bottomrule
  \end{tabular}
\end{table}

\subsection{Algorithmic Trading  of Emission Certificates {\sf (AppCO2)}}
TODO

\lipsum[5]


\section{Case Studies in Water Resource Management}
TODO

\lipsum[6]

\subsection{Fill-level Prediction for Stormwater Tanks {\sf (AppStorm)}}
The main goal of this study is the prediction of fill levels in stormwater overflow tanks based on
rainfall data in order to implement predictive control of water drain rate. Such predictions are of
immense practical utility in preventing costly and damaging over- or under-loading of the sewage
system connected to these stormwater overflow tanks. The task of predicting the current fill levels
from the past rain data alone---not using past fill levels---is rather challenging since the hidden
state of the surrounding soil influences the impact of rain in a nonlinear
fashion~\citep{Bart08c,Kone08b}.

\paragraph{Objective Function}
The quality of a fill level predictor is measured as the root mean squared error (\RMSE) between
true fill level and predicted fill level on the test dataset defined in Sec.~\ref{sec:testdata}:
\begin{equation}\label{eq:rmse}
  \RMSE(\mathsf{y}_{\mathit{pred}}, \mathsf{y}_{\mathit{real}}) := %
    \sqrt{\mathit{mean}[(\mathsf{y}_{\mathit{pred}} - \mathsf{y}_{\mathit{real}})^2]}
\end{equation}
A fill-level predictor is a function from a rainfall time series to a scalar fill level
prediction. When predicting the fill level at time $t$, a fill level predictor has the rainfall time
series up to time $t$ available as input. A time series of predicted fill levels is obtained by
iteratively applying a fill level predictor to a time series of rainfall data.

\paragraph{Test Data}
TODO update ranges

Training and test time series data for this study consist of $25,344$ data records. This data
comprises measurements of the current fill level and the current rainfall at a real stormwater tank
in Germany. These measurements were taken every 5 minutes, ranging from April, 21th 2007 (00:00
a.m.) to July, 17th 2007 (11:55 p.m.). We divided this dataset into a training dataset, ranging from
April, 21th 2007 (00:00 a.m.) to April, 28th 2007 (00:00 a.m.) and a test dataset, ranging from
April, 28th 2007 (00:05 a.m.) to July, 17th 2007 (11:55 p.m.). The results presented in the
following are based on the test dataset, while all methods were trained on the training dataset. The
training dataset consists of a short but balanced sample of dry and rainy days.

Making predictions on the training dataset had to be embedded into the fitness function of GP,
therefore this dataset is comparatively small. We used the same training dataset in each case study
to keep results comparable. While fill levels responds differently to rain depending on season, the
response stays very stable within a season. This is why the size of our training dataset should not
pose a problem to the methods under study, and also why our test dataset spans the late-spring/early
summer season, but not more. In practice, the methods under study would be retrained for each
season.

\subsection{Prediction of NH4N in WTP Inflow {\sf (AppNH4N)}}
TODO

\lipsum[8]

\subsection{Prediction of Chemical Oxygen Demand in WTP Inflow {\sf (AppO2)}}
TODO

\lipsum[9]
