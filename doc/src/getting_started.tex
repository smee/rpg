% getting_started.tex

\chapter{Getting Started}
This chapter provides a short real-world example of performing symbolic regression with \RGP.

\section{Problem Definition}

\subsection{Function Definition}

At first you need to define a fitting problem-function.
As an example you can use the locus of a mathematical pendulum.

\colorbox{lightgray}{mathematicalPendulum <- function(A = 1, g = 9.81, l = 0.1 phi = pi) \{}

\colorbox{lightgray}{omega <- sqr(g/l)}

\colorbox{lightgray}{function(t) A * cos(omega*t*phi)\}} 

This creates a function with the arguments amplitude(A), gravity(g), length(l) and offset(phi).


\subsection{Fitness Function}
The fitness function is used to calculate an error measure.

To define the fitness function:

\colorbox{lightgray}{mp1 <- mathematicalPendulum()}

\colorbox{lightgray}{mp1fitness <- makeFunctionFitnessFunction(mp1, 0, 10, steps = 512, indsizelimit = 16)}

{\bf mp1} is the function as defined in the step before, {\bf o} and {\bf 10} are the start and the 
end of the sequence of fitness cases.The argument{\bf steps} gives the number of steps in the sequence.
{\bf indsizelimit} defines a limit to the size of an individual,
if the limit is exceeded the individual will receive a fitness of inifinity value. 

\subsection{}








