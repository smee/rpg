% getting_started.tex

\chapter{Getting Started}
This chapter provides a short real-world example of performing symbolic regression with \RGP.


\section{Problem Definition}

\subsection{Function Definition}
At first you need to define a fitting problem-function.
As an example you can use the locus of a mathematical pendulum.

\colorbox{lightgray}{mathematicalPendulum <- function(A = 1, g = 9.81, l = 0.1 phi = pi) \{}

\colorbox{lightgray}{omega <- sqr(g/l)}

\colorbox{lightgray}{function(t) A * cos(omega*t*phi)\}} 

This creates a function with the arguments amplitude(A), gravity(g), length(l) and offset(phi).

\subsection{Fitness Function}
The fitness function is used to calculate an error measure.

To define the fitness function:

\colorbox{lightgray}{mp1 <- mathematicalPendulum()}

\colorbox{lightgray}{mp1fitness <- makeFunctionFitnessFunction(mp1, 0, 10, steps = 512, indsizelimit = 16)}

{\bf mp1} is the function as defined in the step before, {\bf o} and {\bf 10} are the start and the 
end of the sequence of fitness cases.The argument{\bf steps} gives the number of steps in the sequence.
{\bf indsizelimit} defines a limit to the size of an individual,
if the limit is exceeded the individual will receive a fitness of inifinity value. 

\subsection{An Example for the Listings Package} % TODO nur ein Beispiel, bitte spaeter entfernen!
Listing \ref{lstExample} shows an example on how to integrate \R code listings into \LaTeX~documents
using the nice {\sf Listings} package\footnote{See
  \url{ftp://ftp.tex.ac.uk/tex-archive/macros/latex/contrib/listings/listings.pdf} for further
  information.}. As you might have noticed, the last sentence also demonstrates how to refer to
listings via the {\tt label}/{\tt ref} pair of \LaTeX~commands.

\begin{lstlisting}[caption = {Some Example \R Code}, label = lstExample]
makePendulum <- function(A = 1, g = 9.81,
                         l = 0.1, phi = pi) {
  omega <- sqr(g / l)
  function(time) A * cos(omega * time * phi)
}
\end{lstlisting}

The {\sf Listings} package also supports typesetting of small code snippets like
\lstinline!gpModel$fitness! through its {\tt lstinline} command. Please note the slightly peculiar way
the snippets are delimited in the \LaTeX~source code. You can use any character that does not occur
in the snippet as a delimiter. We are using the {\tt !} character in this example.
